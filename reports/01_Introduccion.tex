\section{Introducción}
En el siguiente informe se presenta el trabajo realizado durante el período de Trabajo Dirigido con el tema de ``Desempeño de métodos para estimación de períodos''. Actualmente para ALeRCE, específicamente para estrellas variables y períodos, se tienen distintos métodos posibles para la estimación de los períodos de dichas estrellas, dentro de los cuales el método MHAOV es el que actualmente se tiene implementado debido a su buen desempeño. Estos métodos de estimación están implementados en la librería P4J y si bien, se cuenta con evidencia para dar con una preferencia hacia el método MHAOV, no se ha realizado un análisis exhaustivo tomando en cuenta las curvas dobladas con los períodos estimados. Entonces, para este trabajo se realizará un estudio sobre el desempeño de cada uno de estos métodos para distintas clases y número de detecciones utilizadas para la estimación, además de analizar también las etiquetas actuales que se encuentran disponibles. El informe se estructura describiendo la metodología de trabajo utilizada, detallando la base de datos y pasos utilizados para el análisis, seguido de los resultados y análisis de estos, dificultades encontradas y finalmente conclusiones y propuesta para trabajo a futuro. Finalmente, los principales objetivos del trabajo son los siguientes:

	\begin{itemize}
	    \item Familiarizarse con las herramientas existentes para la estimación de períodos (librería P4J). 
	    \item Analizar los períodos de las etiquetas disponibles.
		\item Analizar el desempeño de los distintos métodos para distintas clases y número de muestras.
		\item Comparación del tiempo de cómputo para cada caso.
	\end{itemize}