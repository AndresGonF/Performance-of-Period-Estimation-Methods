% Template:     Informe LaTeX
% Documento:    Archivo principal
% Versión:      7.4.1 (01/09/2021)
% Codificación: UTF-8
%
% Autor: Pablo Pizarro R.
%        pablo@ppizarror.com
%
% Manual template: [https://latex.ppizarror.com/informe]
% Licencia MIT:    [https://opensource.org/licenses/MIT]

% CREACIÓN DEL DOCUMENTO
\documentclass[letterpaper,oneside]{article}

% INFORMACIÓN DEL DOCUMENTO
\def\titulodelinforme {Informe de Trabajo Dirigido}
\def\subtituloinforme {Desempeño de métodos para estimación de períodos}
\def\temaatratar {Desempeño de métodos para estimación de períodos}

\def\autordeldocumento {Andrés González Fuentes}
\def\nombredelcurso {Trabajo Dirigido}
\def\codigodelcurso {EL7850-1}

\def\nombreuniversidad {Universidad de Chile}
\def\nombrefacultad {Facultad de Ciencias Físicas y Matemáticas}
\def\departamentouniversidad {Departamento de Ingeniería Eléctrica}
\def\imagendepartamento {departamentos/die}
\def\imagendepartamentoparams {height=1.57cm}
\def\localizacionuniversidad {Santiago de Chile}

% INTEGRANTES, PROFESORES Y FECHAS
\def\tablaintegrantes {
\begin{tabular}{ll}
	Integrantes:
	& \begin{tabular}[t]{l}
		Andrés González F. \\
	\end{tabular} \\
	Tutores:
	& \begin{tabular}[t]{l}
		Pablo Estevez V. \\
		Ignacio Reyes J.
	\end{tabular} \\
	\multicolumn{2}{l}{Fecha de entrega: \today} \\
	\multicolumn{2}{l}{\localizacionuniversidad}
\end{tabular}}{
}

% IMPORTACIÓN DEL TEMPLATE
\input{template}

% INICIO DE PÁGINAS
\begin{document}
	
% PORTADA
\templatePortrait

% CONFIGURACIÓN DE PÁGINA Y ENCABEZADOS
\templatePagecfg

% RESUMEN O ABSTRACT
% \begin{resumen}
% 	\lipsum[1] % Párrafo ejemplo, se puede borrar
% \end{resumen}

% TABLA DE CONTENIDOS - ÍNDICE
\templateIndex

% CONFIGURACIONES FINALES
\templateFinalcfg

% ======================= INICIO DEL DOCUMENTO =======================

% \input{example} % Ejemplo, se puede borrar
\section{Introducción}
En el siguiente informe se presenta el trabajo realizado durante el período de Trabajo Dirigido con el tema de ``Desempeño de métodos para estimación de períodos''. Actualmente para ALeRCE, específicamente para estrellas variables y períodos, se tienen distintos métodos posibles para la estimación de los períodos de dichas estrellas, dentro de los cuales el método MHAOV es el que actualmente se tiene implementado debido a su buen desempeño. Estos métodos de estimación están implementados en la librería P4J y si bien, se cuenta con evidencia para dar con una preferencia hacia el método MHAOV, no se ha realizado un análisis exhaustivo tomando en cuenta las curvas dobladas con los períodos estimados. Entonces, para este trabajo se realizará un estudio sobre el desempeño de cada uno de estos métodos para distintas clases y número de detecciones utilizadas para la estimación, además de analizar también las etiquetas actuales que se encuentran disponibles. El informe se estructura describiendo la metodología de trabajo utilizada, detallando la base de datos y pasos utilizados para el análisis, seguido de los resultados y análisis de estos, dificultades encontradas y finalmente conclusiones y propuesta para trabajo a futuro. Finalmente, los principales objetivos del trabajo son los siguientes:

	\begin{itemize}
	    \item Familiarizarse con las herramientas existentes para la estimación de períodos (librería P4J). 
	    \item Analizar los períodos de las etiquetas disponibles.
		\item Analizar el desempeño de los distintos métodos para distintas clases y número de muestras.
		\item Comparación del tiempo de cómputo para cada caso.
	\end{itemize}
\newpage
\section{Metodología de trabajo}

    \subsection{Base de datos}
    
    Para realizar la evaluación de desempeño de los métodos de estimación de período que utiliza, se utilizará una base de datos correspondiente a las detecciones de datos reales del observatorio Zwicky Transient Facility (ZTF). Debido a que el trabajo esta acotado solo a estrellas periódicas, se seleccionarán solo a aquellas que lo sean y de estas, se utilizarán las etiquetas disponibles de diversos catálogos que las categoriza en alguna de las clases mencionadas en la sección anterior y que cuentan con sus propios períodos, los cuales también será evaluados. La distribución de clases y los catálogos utilizados se presentan resumidos en la Tabla \ref{tab:nClases} y la Tabla \ref{tab:nCata}, respectivamente.\\
    
    \begin{multicols}{2}
    \centering
        \begin{tabular}{lr}
        \toprule
        {} &  \textbf{classALeRCE} \\
        \midrule
        \textbf{EB/EW         } &        47541 \\
        \textbf{RRL           } &        21483 \\
        \textbf{LPV           } &         5119 \\
        \textbf{DSCT          } &         1071 \\
        \textbf{Periodic-Other} &          867 \\
        \textbf{Ceph          } &          786 \\
        \bottomrule
        \end{tabular}
        \captionof{table}{Cantidad de objetos por clases.}
        \label{tab:nClases}

    \centering
        \begin{tabular}{lr}
        \toprule
        {} &  \textbf{source} \\
        \midrule
        \textbf{CRTSnorth} &   46123 \\
        \textbf{ASASSN   } &   13668 \\
        \textbf{GAIADR2VS} &    8140 \\
        \textbf{CRTSsouth} &    5337 \\
        \textbf{LINEAR   } &    3599 \\
        \bottomrule
        \end{tabular}
        \captionof{table}{Cantidad de objetos por catálogo.}
        \label{tab:nCata}
    \end{multicols}
    
    
    \subsection{Selección de objetos}
    
     De la base de datos, solo se seleccionaron 100 objetos por clase que cumplen con un mínimo de muestras en cada banda (60 en este caso) y cuyo error sea menor o igual a 1, como muestra la Tabla \ref{tab:selectedData}.

    \begin{table}
    \centering
    \caption{Resumen de la base de datos a utilizar.}
    \label{tab:selectedData}
    \scalebox{0.85}{%
        \begin{tabular}{lllll}
        \toprule
        {} & N° objs & N° samples & N° min-max samples (g) & N° min-max samples (r) \\
        \midrule
        \textbf{RRL } &     100 &      28468 &               65 - 548 &               61 - 516 \\
        \textbf{Ceph} &     100 &      39862 &               68 - 631 &               62 - 794 \\
        \textbf{LPV } &     100 &      39082 &               71 - 491 &               60 - 640 \\
        \textbf{DSCT} &     100 &      28397 &               62 - 406 &               60 - 413 \\
        \textbf{EB  } &     100 &      27581 &               60 - 476 &               61 - 690 \\
        \bottomrule
        \end{tabular}}
    \end{table}
    
    \subsection{Estimación de períodos}
    
    Luego, para contar con la estimación de los períodos, se hará uso del paquete de python P4J \cite{p4j} y también astropy, a partir del cual se generarán los períodogramas de cada banda o multibanda según venga el caso. Posteriormente se estimará el mejor período con cada uno de los 6 métodos implementados en P4J y Lomb-Scargle en astropy, junto con su tiempo de cómputo, y se almacenarán en un archivo .csv para tenerlos siempre disponibles para un DataFrame.
    
    
    \subsection{Calificación de curvas dobladas}
    
	Una vez realizadas las estimaciones de período, cada período es testeado realizando una inspección visual de las curvas dobladas (Figura \ref{img:curvaDoblaje}) para los objetos respectivos. 
    
	\insertimage[\label{img:curvaDoblaje}]{curvaDoblaje.PNG}{scale=0.7}{Ejemplo de curva de luz (izquierda) con su respectiva curva doblada para un período dado (derecha).}    
    
    Finalmente, para evaluar la curva doblada se definen una serie de criterios  a los cuales se les asignará un número para distinguirlos:
    
    \begin{itemize}
        \item \textbf{Ok (1):} Se considerarán correctos (``Ok'') aquellos períodos que den una curva doblada de acuerdo a su clase (ver Anexo \ref{an:ok}).
        
        \item \textbf{Multiply (2 y 3 para EB):} Se considerarán como múltiplos (“Multiply”) aquellos períodos cuya curvadoblada sea un múltiplo del período real, por ejemplo cuando se ve más de unperíodo graficado (ver Anexo \ref{an:mult}).
        
        \item \textbf{Differs (0):} Se considerarán como erróneos (“Differs”) aquellos períodos cuya curva dobladano tenga una curva clara o no no corresponda a su clase respectiva (ver Anexo \ref{an:differs}).
        
        \item \textbf{Half (EB, 2):} Considerando solo binarias eclipsantes, aquellos períodos cuya curva doblada seala mitad de la curva real, serán considerados como “Half”. Se hace estadiferencia con “Multiply” por ser un error común dentro de esta clas que serepite para la mitad del período (ver Anexo \ref{an:half}).
    \end{itemize}
    

\newpage
\section{Resultados y análisis}

    \subsection{Catálogo}
    
    A partir de la evaluación de las curvas dobladas de las etiquetas se puede realizar un análisis en base a las clases que contiene (Tabla \ref{tab:catClass}) y al catálogo al que pertenecen (Tabla \ref{tab:catCat}). Este último fue obtenido promediando el hitrate entre cada clase para un catálogo determinado
	\begin{multicols}{2}

	\begin{table}
		\centering
		\caption{Resumen de etiquetas por clases.}
		\begin{tabular}{cc}
			\hline
			\textbf{Clase}      & \textbf{[\%] de acierto} \bigstrut\\
			\hline
			\textbf{RLL}        & 0.99 \bigstrut[t]\\
			\textbf{Cefeidas}   & 0.82 \\
			\textbf{LPV}        & 0.80  \\
			\textbf{DSCT}       & 0.89 \\
			\textbf{EB}         & 0.87 \bigstrut[b] \\
			\hline
		\end{tabular}
		\label{tab:catClass}
	\end{table}
	
		\begin{table}
		\centering
		\caption{Resumen de etiquetas por catálogo}
		\begin{tabular}{cc}
			\hline
			\textbf{Clase}      & \textbf{[\%] de acierto} \bigstrut\\
			\hline
			\textbf{CRTSnorth}  & 0.94 \bigstrut[t]\\
			\textbf{GAIADR2VS}  & 0.87 \\
			\textbf{ASASSN}     & 0.89 \\
			\textbf{LINEAR}     & 0.76 \\
			\textbf{CRTSsouth}  & 1.00 \bigstrut[b] \\
			\hline
		\end{tabular}
		\label{tab:catCat}
	\end{table}
	\end{multicols}    
	
	De las Tablas \ref{tab:catClass} y \ref{tab:catCat} se puede ver que en general, los períodos de las etiquetas sirven al menos para el 80\% de cada clase, y los catálogos tienen buena representatividad, con excepción de LINEAR que solo llega al 76\%.	Esto último se debe intrínsecamente a que Cefeidas y LPV solo tienen dos objetos de este catálogo dentro de los datos utilizados, de los cuales uno tiene un período incorrecto, lo que disminuye a la mitad el hit rate y provoca la baja al calcular el promedio entre clase.

\newpage
    \subsection{Hit-rate}
        \subsubsection{RRL}
    	\insertimage[\label{img:hitrateRRL}]{hitrate/RRL_hitrate4.png}{scale=0.35}{Comparación de hit rate de cada método entre bandas g, r y multibanda MHAOV para RRL.}
    	
    	\begin{itemize}
    	    \item \textbf{Single band:} De los hit-rate en la Figura \ref{img:hitrateRRL} se tiene que para una sola banda, la forma de la curva es similar entre las bandas g y r. Además, se ve que algunos métodos tienen desempeños similares entre ellos, como QME y QMIEU, y LS con MHAOV (casi idénticos). Se observa que la mejora en hit-rate comienza a detenerse desde las 40 muestras y en términos de desempeño general, la banda g es mejor respecto a la banda r. Se destaca que LKSL y PDM1 son notablemente inferiores al resto de los métodos.
    	    
    	    \item \textbf{Multi band:} Observando el hit-rate en multibanda de la Figura \ref{img:hitrateRRL}, lo primero que se puede notar es que ya inicialmente con 10 muestras, el desempeño es mejor en que single band y luego ya con solo 20 muestras se alcanza un hit-rate de 0.78. También se puede hacer la diferencia respecto al single band, que ahora las mejoras con mayor número de muestras comienza a descender desde las 30 muestras en vez de 40. Finalmente, se observa que el hit-rate final es similar al alcanzado por los métodos cuadráticos en la banda g.
    	\end{itemize}
    	
	    \subsubsection{Cefeidas}
    	\insertimage[\label{img:hitrateCeph}]{hitrate/Ceph_hitrate4.png}{scale=0.35}{Comparación de hit rate de cada método entre bandas g, r y multibanda MHAOV para Cefeidas.}
    	
    	\begin{itemize}
    	    \item \textbf{Single band:} De los hit-rate en la Figura \ref{img:hitrateCeph} se tiene que para una sola banda, la forma de la curva es similar entre las bandas g y r. Inicialmente se tiene un hit-rate de 0.20 para algunos métodos. La diferencia en desempeño entre cada método es menor respecto a las vistas para RRL, estos es, las curvas de cada método están mas juntas entre sí. Para la banda g se tiene un desempeño ligeramente mayor a la banda r, y al final con 50 muestras se puede ver esto con mayor facilidad. En este punto, todos los métodos para la banda g parecen converger al mismo hit-rate, que es superior al alcanzado por los métodos en la banda r, en donde además éstos se encuentran más separados a diferencia de la banda g. Finalmente se destaca que la diferencia entre métodos disminuye a mayor número de muestras y que las curvas para las cefeidas tienen mayor pendiente de mejora que para las RRL.
    	    
    	    \item \textbf{Multi band:} Observando el hit-rate en multibanda de la Figura \ref{img:hitrateCeph}, se tiene un desempeño inicial de 0.6 para 10 muestras y sobre 0.8 desde las 20 muestras. Luego de esto, el desempeño va mejorando con un comportamiento aparentemente lineal desde las 30 muestras, para así llegar con un hit-rate bastante cercano a 1.0 para las 50 muestras.
    	\end{itemize}
    	
	    \subsubsection{LPV}
    	\insertimage[\label{img:hitrateLPV}]{hitrate/LPV_hitrate4.png}{scale=0.35}{Comparación de hit rate de cada método entre bandas g, r y multibanda MHAOV para LPV.}
    	

    	\begin{itemize}
    	    \item \textbf{Single band:} De los hit-rate en la Figura \ref{img:hitrateLPV}, se tiene que a diferencia de las RRL y cefeidas, existen diferencias notables entre la mayoría de los métodos y ya no se comparte la misma forma de la curva, salvo para los pares LS - MHAOV y QME - QMIEU que se comportan de manera similar solo entre sí. Luego, se tiene que ahora no son los métodos cuadráticos los que introducen la mayor mejora en hit-rate, sino que son los métodos de primero orden, en donde LS y MHAOV son quienes tienen el mejor desempeño. También se puede notar que la banda r tiene un mejor desempeño que la banda g. El máximo hit-rate alcanzado es un poco inferior a 0.8.
    	    
    	    \item \textbf{Multi band:}  Observando el hit-rate en multibanda de la Figura \ref{img:hitrateLPV}, se encuentra una mejora notable respecto a los métodos en single band, en donde con solo 10 muestras se tiene un hit-rate superior a 0.5. Luego, se puede notar que rápidamente el método llega a una cota de aproximadamente 0.8 desde las 20 muestras, con un desempeño final con una leve mejora respecto al mejor hit-rate en la banda r.
    	\end{itemize}
    	

	    \subsubsection{DSCT}
    	\insertimage[\label{img:hitrateDSCT}]{hitrate/DSCT_hitrate4.png}{scale=0.35}{Comparación de hit rate de cada método entre bandas g, r y multibanda MHAOV para DSCT.}
    	
    	\begin{itemize}
    	    \item \textbf{Single band:} De los hit-rate en la Figura \ref{img:hitrateDSCT}, lo primero que se puede notar es que el hit-rate va mejorando de una forma aproximadamente lineal para todos los métodos, exceptuando LKSL y PDM1, en donde tanto para banda g como para la banda r aparentemente se alcanza una cota cerca de 0.2. En general, se tiene que el desempeño va aumentando lentamente a mayor número de muestras y para las 50 muestras, no se observa ninguna cota y llega hasta los 0.6. Esto indica que el hit-rate podría seguir mejorando con un mayor número de muestras para la estimación del período, salvo para LKSL y PDM1. Finalmente, los mejores resultados se dan utilizando los métodos cuadráticos y no parece haber una mejora en desempeño entre una banda y la otra.
    	    
    	    \item \textbf{Multi band:}  Observando el hit-rate en multibanda de la Figura \ref{img:hitrateDSCT}, se tiene una mejora inicial a las 10 muestras que es inferior a la vista para las otras clases. Luego, se observa una rápida mejora en desempeño hasta las 30 muestras, momento en la que esta mejora se vuelva más lenta. Finalmente, llegando a las 50 muestras se tiene un hit-rate final que no supera los 0.6 y que no supera el mejor hit-rate en single band, sin embargo, se ve que este puede seguir mejorando pero con un ritmo más lento que para single band.
    	\end{itemize}    	





        \subsubsection{EB}
    	\insertimage[\label{img:hitrateEB}]{hitrate/EB_hitrate4.png}{scale=0.35}{Comparación de hit rate de cada método entre bandas g, r y multibanda MHAOV para EB.}
    	
    	Observando los hit-rate para la clase EB en la Figura \ref{img:hitrateEB}, se puede ver utilizando una sola banda, los únicos métodos que logran tener un hit-rate superior a 0.2 son LKSL y PDM1. El resto de los métodos tienen un desempeño casi nulo y para el caso multibanda, este es completamente nulo.\\
    	
    	Debido a la particularidad que tiene la clase EB en su estimación de período es que se incluyó un tipo de error distinto que aplica solo para esta clase, que es cuando la mitad del período es estimada. El hit-rate medido tomando en cuenta las estimaciones con este error se puede ver en la Figura \ref{img:hitrateEB_half}.
    	
    	\insertimage[\label{img:hitrateEB_half}]{hitrate/EB_hitrate4_half.png}{scale=0.35}{Comparación de hit rate medido con el error ``half'' de cada método entre bandas g, r y multibanda MHAOV para EB.}  
    	
    	\begin{itemize}
    	    \item \textbf{Single band:} De los hit-rate en la Figura \ref{img:hitrateEB_half}, teniendo de referencia el error de la mitad del período, se puede ver que en general no hay una banda mejor que la otra en desempeño. Luego, en ambas bandas se ve que los métodos LKSL y PDM1 presentan un menor hit-rate que el resto de los métods que son similares entre ellos. Para el caso de la banda g, los métodos (exceptuando LKSL y PDM1) convergen a las 30 muestras en un mismo hit-rate para luego ir mejorando de manera aparentemente lineal y de entre estos métodos, QME presenta una ligera mejora con una mayor pendiente. Para el caso de la banda r, los métodos (exceptuando LKSL y PDM1) no convergen en ningún momento pero durante la mayor parte del tiempo se encuentran poco separados entre sí. Finalmente se observa que el desempeño puede seguir mejorando más lentamente con un mayor número de muestras.
    	    
    	    \item \textbf{Multi band:}  Observando el hit-rate en multibanda de la Figura \ref{img:hitrateEB_half}, se tiene un desempeño inicial de casi 0.3, mejorando respecto al caso single band. Luego, para la mayoría de las distintas cantidades de muestras el desempeño en multi band es superior al caso single band, hasta las 50 muestras en donde él desempeño es similar al mejor hit-rate en la banda g.
    	\end{itemize}    	
    	
    	\insertimage[\label{img:hitrateEB_ok_half}]{hitrate/EB_hitrate4_ok_half.png}{scale=0.35}{Comparación de hit rate medido con la suma entre estimaciones correctas y con error ``half'', de cada método entre bandas g, r y multibanda MHAOV para EB.}
    	
    	\begin{itemize}
    	    \item \textbf{Single band:} De los hit-rate en la Figura \ref{img:hitrateEB_ok_half}, se puede ver que hasta las 20 muestras, no se introducen mejoras al considerar estimaciones correctas y con la mitad del período. Sin embargo, a partir de las 30 muestras si existen notables mejoras para los métodos LKSL y PDM1, en donde rápidamente su hit-rate sube hasta llegar a valores levemente superiores a 0.6 para las 50 muestras, convirtiéndose en los mejores métodos de estimación para esta clase. Para el resto de los métodos no existen grandes cambios.
    	    
    	    \item \textbf{Multi band:}  Observando el hit-rate en multibanda de la Figura \ref{img:hitrateEB_ok_half}, no hay cambio alguno al considerar tanto el error con la mitad del período y los períodos correctos. Esto se debe a que el hit-rate con el período correcto es totalmente nulo. Además, el hit-rate final es inferior al alcanzado por la banda r en single band que supera los 0.6.
    	\end{itemize}    	
    	
    	
    	
\newpage
	\subsection{Tiempo de cómputo}
	
    De las Figuras \ref{img:timeRRL}, \ref{img:timeCeph}, \ref{img:timeLPV}, \ref{img:timeDSCT} y \ref{img:timeEB} se puede ver como varía el tiempo de cómputo con el número de muestras entre los distintos métodos. Dado que los resultados en general son bastantes similares entre las clases, se muestran todos estos en listado para dar con un análisis general posteriormente.
    
    % \subsubsection{RRL}
    	\insertimage[\label{img:timeRRL}]{time/RRL_time3.png}{scale=0.3}{Comparación de tiempo de cómputo de cada método entre bandas g, r y multibanda MHAOV para RRL.}
    	
    % 	De los tiempos de cómputo en escala logarítmica en la Figura \ref{img:timeRRL}, se tiene que entre ambas bandas no hay diferencias. Los métodos cuadráticos rápidamente crecen en tiempo de cómputo con una mayor pendiente que los métodos de primer órden, debido a la complejidad cuadrática. Para el caso multibanda se tiene un tiempo intermedio entre los métodos cuadráticos y de primer orden, y una curva que parece presentar una cota a mayor número de muestras.
	
    	
    % 	\subsubsection{Cefeidas}
		\insertimage[\label{img:timeCeph}]{time/Ceph_time3.png}{scale=0.3}{Comparación de tiempo de cómputo de cada método entre bandas g, r y multibanda MHAOV para Cefeidas.}
		
    % 	De los tiempos de cómputo en escala logarítmica en la Figura \ref{img:timeCeph}, se tiene que entre ambas bandas no hay diferencias. Los métodos cuadráticos rápidamente crecen en tiempo de cómputo con una mayor pendiente que los métodos de primer órden, debido a la complejidad cuadrática. Para el caso multibanda se tiene un tiempo intermedio entre los métodos cuadráticos y de primer orden, y una curva que parece presentar una cota a mayor número de muestras.		
		
    	
    % 	\subsubsection{LPV}
    	\insertimage[\label{img:timeLPV}]{time/LPV_time3.png}{scale=0.3}{Comparación de tiempo de cómputo de cada método entre bandas g, r y multibanda MHAOV para LPV.}	
    	
    % 	De los tiempos de cómputo en escala logarítmica en la Figura \ref{img:timeLPV}, se tiene que entre ambas bandas no hay diferencias. Los métodos cuadráticos rápidamente crecen en tiempo de cómputo con una mayor pendiente que los métodos de primer órden, debido a la complejidad cuadrática. Para el caso multibanda se tiene un tiempo intermedio entre los métodos cuadráticos y de primer orden, y una curva que parece presentar una cota a mayor número de muestras.		
    	
    	
    	
    % 	\subsubsection{DSCT}
        \insertimage[\label{img:timeDSCT}]{time/DSCT_time3.png}{scale=0.3}{Comparación de tiempo de cómputo de cada método entre bandas g, r y multibanda MHAOV para DSCT.}	
        
    % 	De los tiempos de cómputo en escala logarítmica en la Figura \ref{img:timeDSCT}, se tiene que entre ambas bandas no hay diferencias. Los métodos cuadráticos rápidamente crecen en tiempo de cómputo con una mayor pendiente que los métodos de primer órden, debido a la complejidad cuadrática. Para el caso multibanda se tiene un tiempo intermedio entre los métodos cuadráticos y de primer orden, y una curva que parece presentar una cota a mayor número de muestras.		
        
        
	
	   % \subsubsection{EB}
		\insertimage[\label{img:timeEB}]{time/EB_time3.png}{scale=0.3}{Comparación de tiempo de cómputo de cada método entre bandas g, r y multibanda MHAOV para EB.}
		
    % 	De los tiempos de cómputo en escala logarítmica en la Figura \ref{img:timeDSCT}, se tiene que entre ambas bandas no hay diferencias. Los métodos cuadráticos rápidamente crecen en tiempo de cómputo con una mayor pendiente que los métodos de primer órden, debido a la complejidad cuadrática. Para el caso multibanda se tiene un tiempo intermedio entre los métodos cuadráticos y de primer orden, y una curva que parece presentar una cota a mayor número de muestras.		
    
    
	   % \item Tiempo inicial mayor para 10 muestras que en caso single band.
	   % \item El tiempo de cómputo aumenta hasta llegar a una cota con N° muestras.
    
    \begin{itemize}
        \item \textbf{Single band:} Para las Figuras anteriores, se puede notar que entre las distintas bandas, prácticamente no hay diferencia alguna con cualquier número de muestras. Luego, se tiene que los métodos cuadráticos presentan una mayor pendiente que aquellos métodos de primer orden y además, estos son prácticamente idénticos entre sí. También para los métodos de primer orden se observa que estos difieren levemente entre ellos, en especial entre los métodos LKSL y MHAOV que son bastante similares entre sí. Finalmente, de entre los métodos de primer orden se tiene que LS ocupa mayor tiempo de cómputo, seguido por el par LKSL - MHAOV y PDM1 siendo el más rápido de todos los métodos.
        
        \item \textbf{Multi band:} Observando el tiempo de cómputo en multi banda de todas las clases, se observa que el tiempo inicial para 10 muestras es mayor que en el caso de single band, y para a las 50 muestras es menor que para los métodos cuadráticos en general. Respecto a la pendiente que tiene, se nota que está en un punto intermedio entre la pendiente de los métodos cuadráticos y la de los métodos de primer orden.
    \end{itemize}
		
	
	
	

	

	

	

	
	
\newpage
\section{Dificultades}

    \begin{itemize}
        \item Al inicio del trabajo no se tenía claro los datos a utilizar, dificultando la obtención de curvas correctas. Esto debido a que la lista des las detecciones manejada por ALeRCE cuenta con múltiples features que podían estar o no corregidas, hecho que era desconocido en un comienzo (se tiene que trabajar con los corregidos).
        
        \item Al ir estimando los períodos de los objetos, ocurría que cuando existían observaciones nulas, estas se propagaban dentro de los métodos implementados por la librería P4J, lo cual daba un período final nulo. Esto se debía a un pequeño error en la asignación de variables al limpiar datos nulos dentro de la librería, el cual fue corregido eventualmente.
        
        \item Al ir realizando la inspección visual de la curvas dobladas, con el tiempo se notó que involucra un tiempo no menor debido a la gran cantidad de curvas por ver (100 objetos x 7 métodos x 5 N° muestras distintos x 6 clases = 21000 curvas), lo que además puede involucrar mayor tiempo en caso de tener errores imprevistos.
        
        \item A medida que se iban inspeccionando las curvas dobladas, se noto que existen muestras con un ruido considerablemente grande que desplazaban los límites de los gráficos. Este hecho era más evidente para las LPVs y en consecuencia, se decidió aplicar un filtro inicial al seleccionar los objetos de la base de datos.
    \end{itemize}
\section{Conclusiones y trabajo a futuro}

    \begin{itemize}
        \item Respecto a los resultados obtenidos con los catálogos, se puede decir que son una buena referencia de comparación con un hit-rate de al menos 0.80 para todas las clases. A futuro podría ser interesante hacer el mismo estudio con otros tipos de catálogos para comparar las fuentes y analizar como cambian los porcentajes de acierto.
        \item Es evidente que no todas las clases tienen el mismo desempeño en general. De mejor a peor con 50 muestras se tiene Ceph, RRL, LPV, DSCT, y EB. Luego, viendo cada caso en particular en las clases, se tiene que algunos métodos son mejor aplicables para algunas clases, por lo que para mejorar el desempeño general, una combinación de los métodos a utilizar dependiendo del caso podría ser útil y en ese sentido puede ayudar identificar las fallas específicas de los métodos y también las características de los objetos que fallan con determinados métodos.
        \item Los períodos con LKSL y PDM1 resultan en múltiplos en mayor cantidad. Esto resulta ser beneficioso para EBs y LPVs, y se debe a que en general estos métodos se enfocan más en la forma de las curvas.
        \item El método de LS y MHAOV son bastante parecidos entre sí en términos de hit-rate. Sin embargo, si se toma en cuenta el tiempo de cómputo, el método MHAOB termina siendo mejor.
        \item El tiempo de cómputo depende netamente de la complejidad del algoritmo y no varía mayormente entre clases.
        
        \item Queda pendiente aplicar el análisis con multibandas faltantes (LKSL, PDM1, métodos cuadráti-cos).
        
    \end{itemize}
% ------------------------------------------------------------------------------
% REFERENCIAS (ESTILO BIBTEX), revisar configuración \stylecitereferences
% ------------------------------------------------------------------------------
\clearpage % Salto de página
\begin{references}
	\bibitem{catelan}
	M. Catelan, H. Smith ``Pulsating Stars''. Weinheim Germany: Wiley-VCH, 2015.
	
	\bibitem{phuijse}
	P. Huijse et al ``Robust Period Estimation Using Mutual Information for Multiband Light Curves in the Synoptic Survey Era'' in ApJS, 2018, 236 12.
	
	\bibitem{aov}
	N. Mondrik et al ``A MULTIBAND GENERALIZATION OF THE MULTIHARMONIC ANALYSIS OF VARIANCE PERIOD ESTIMATION ALGORITHM AND THE EFFECT OF INTER-BAND OBSERVING CADENCE ON PERIOD RECOVERY RATE'' in ApJL, 2015, 811 L34.
	
	\bibitem{p4j}
	P. Huijse, P. Protopapas, P. Estévez, P. Zegers, J. Príncipes, ``P4J'', 2016. \\
	\url{https://github.com/phuijse/P4J}

    
\end{references}

% ------------------------------------------------------------------------------
% ANEXO
% ------------------------------------------------------------------------------
\clearpage
\begin{anexo}

\section{Criterios}
    \subsection{Ok}
    \label{an:ok}
	\begin{multicols}{3}

		% Párrafo 1
		\insertimage[\label{img:okRRL}]{criterios/crit_ok-1.PNG}{scale=0.4}{Ejemplo de curva doblada correctamente para RRL.}  
		
		\insertimage[\label{img:okCeph}]{criterios/crit_ok-2.PNG}{scale=0.5}{Ejemplo de curva doblada correctamente para Cefeida.}  		
		
		\insertimage[\label{img:okLPV}]{criterios/crit_ok-3.PNG}{scale=0.5}{Ejemplo de curva doblada correctamente para LPV.}  		

	\end{multicols}
	
	\begin{multicols}{2}
		% Párrafo 2
		\insertimage[\label{img:okDSCT}]{criterios/crit_ok-4.PNG}{scale=0.5}{Ejemplo de curva doblada correctamente para DSCT.}  
		
		\insertimage[\label{img:okEB}]{criterios/crit_ok-5.PNG}{scale=0.5}{Ejemplo de curva doblada correctamente para EB.}  
	\end{multicols}
	
	\subsection{Multiply}
	\label{an:mult}
	\begin{multicols}{3}

		% Párrafo 1
		\insertimage[\label{img:multRRL}]{criterios/crit_mult-1.PNG}{scale=0.4}{Ejemplo de curva doblada con período múltiplo para RRL.}  
		
		\insertimage[\label{img:multCeph}]{criterios/crit_mult-2.PNG}{scale=0.5}{Ejemplo de curva doblada con período múltiplo para Cefeida.}  		
		
		\insertimage[\label{img:multLPV}]{criterios/crit_mult-3.PNG}{scale=0.5}{Ejemplo de curva doblada con período múltiplo para LPV.}  		

	\end{multicols}
	
	\begin{multicols}{2}
		% Párrafo 2
		\insertimage[\label{img:multDSCT}]{criterios/crit_mult-4.PNG}{scale=0.5}{Ejemplo de curva doblada con período múltiplo para DSCT.}  
		
		\insertimage[\label{img:multEB}]{criterios/crit_mult-5.PNG}{scale=0.5}{Ejemplo de curva doblada con período múltiplo para EB.}  
	\end{multicols}	
	
	
	\subsection{Differs}
	\label{an:differs}
	\begin{multicols}{3}

		% Párrafo 1
		\insertimage[\label{img:wrongRRL}]{criterios/crit_wrong-1.PNG}{scale=0.4}{Ejemplo de curva doblada incorrectamente para RRL.}  
		
		\insertimage[\label{img:wrongCeph}]{criterios/crit_wrong-2.PNG}{scale=0.5}{Ejemplo de curva doblada incorrectamente para Cefeida.}  		
		
		\insertimage[\label{img:wrongLPV}]{criterios/crit_wrong-3.PNG}{scale=0.5}{Ejemplo de curva doblada incorrectamente para LPV.}  		

	\end{multicols}
	
	\begin{multicols}{2}
		% Párrafo 2
		\insertimage[\label{img:wrongDSCT}]{criterios/crit_wrong-4.PNG}{scale=0.5}{Ejemplo de curva doblada incorrectamente para DSCT.}  
		
		\insertimage[\label{img:wrongEB}]{criterios/crit_wrong-5.PNG}{scale=0.5}{Ejemplo de curva doblada incorrectamente para EB.}  
	\end{multicols}	
	
\newpage
	\subsection{Half}
	\label{an:half}
	\begin{multicols}{2}

		% Párrafo 1
		\insertimage[\label{img:halfEB1}]{criterios/EB_half-1.PNG}{scale=0.5}{Ejemplo de curva doblada con la mitad del período para EB.}  
		
		\insertimage[\label{img:halfEB2}]{criterios/EB_half-2.PNG}{scale=0.5}{Ejemplo de curva doblada con la mitad del período para EB.}  		
		
		% Párrafo 2
		\insertimage[\label{img:halfEB3}]{criterios/EB_half-3.PNG}{scale=0.5}{Ejemplo de curva doblada con la mitad del período para EB.}  
		
		\insertimage[\label{img:halfEB4}]{criterios/EB_half-4.PNG}{scale=0.5}{Ejemplo de curva doblada con la mitad del período para EB.}  
	\end{multicols}		
	
	
\newpage	
\section{Desglose de resultados}

    \subsection{Catálogo}
    
    \begin{table}
        \centering
        \caption{Resumen de resultados de catálogos por clase y fuente.}
        \label{tab:catComplete}
        \scalebox{.8}{%
        \begin{tabular}{llrrrrr}
            \toprule
               & \textbf{source} &  \textbf{ASASSN} &  \textbf{CRTSnorth} &  \textbf{GAIADR2VS} &  \textbf{LINEAR} &  \textbf{CRTSsouth} \\
            {} & \textbf{catalog} &         &            &            &         &            \\
            \midrule
            \textbf{RRL} & \textbf{Right} &      13 &         33 &       24.0 &      29 &        NaN \\
               & \textbf{Multiply} &       1 &          0 &        0.0 &       0 &        NaN \\
               \hline
            \textbf{Ceph} & \textbf{Wrong} &       8 &          5 &        NaN &       1 &        NaN \\
               & \textbf{Right} &      59 &         22 &        NaN &       1 &        NaN \\
               & \textbf{Multiply} &       2 &          2 &        NaN &       0 &        NaN \\
               \hline
            \textbf{LPV} & \textbf{Wrong} &       0 &          2 &       14.0 &       1 &        NaN \\
               & \textbf{Right} &      10 &          7 &       62.0 &       1 &        NaN \\
               & \textbf{Multiply} &       0 &          0 &        3.0 &       0 &        NaN \\
               \hline
            \textbf{DSCT} & \textbf{Wrong} &       4 &          4 &        NaN &       3 &        NaN \\
               & \textbf{Right} &      54 &         23 &        NaN &      12 &        NaN \\
               \hline
            \textbf{EB} & \textbf{Wrong} &       0 &          1 &        NaN &       0 &        0.0 \\
               & \textbf{Right} &      23 &         70 &        NaN &       3 &        1.0 \\
               & \textbf{Multiply} &       0 &          2 &        NaN &       0 &        0.0 \\
            \bottomrule
        \end{tabular}}
    \end{table}
    
    
    \subsection{RRL - Single band}
    
    
\begin{table}
\centering
\caption{Resumen de resultados RRL single band.}
\label{anexo:RRLsb}
\scalebox{.48}{
\begin{tabular}{lrrrrrrrrrrrrrrr}
\toprule
{} & \multicolumn{3}{l}{10 samples:} & \multicolumn{3}{l}{20 samples:} & \multicolumn{3}{l}{30 samples:} & \multicolumn{3}{l}{40 samples:} & \multicolumn{3}{l}{50 samples:} \\
{} &     Differs &    Ok & Multiply &     Differs &    Ok & Multiply &     Differs &    Ok & Multiply &     Differs &    Ok & Multiply &     Differs &    Ok & Multiply \\
\midrule
\textbf{catalog} &        0.00 &  0.99 &     0.01 &        0.00 &  0.99 &     0.01 &        0.00 &  0.99 &     0.01 &        0.00 &  0.99 &     0.01 &        0.00 &  0.99 &     0.01 \\
\textbf{LS\_g   } &        0.93 &  0.06 &     0.01 &        0.60 &  0.40 &     0.00 &        0.24 &  0.76 &     0.00 &        0.17 &  0.83 &     0.00 &        0.17 &  0.83 &     0.00 \\
\textbf{LS\_r   } &        0.95 &  0.05 &     0.00 &        0.66 &  0.34 &     0.00 &        0.35 &  0.64 &     0.01 &        0.25 &  0.75 &     0.00 &        0.22 &  0.78 &     0.00 \\
\textbf{LKSL\_g } &        1.00 &  0.00 &     0.00 &        0.62 &  0.36 &     0.02 &        0.33 &  0.61 &     0.06 &        0.17 &  0.70 &     0.13 &        0.12 &  0.70 &     0.18 \\
\textbf{LKSL\_r } &        0.99 &  0.01 &     0.00 &        0.63 &  0.31 &     0.06 &        0.37 &  0.57 &     0.06 &        0.26 &  0.67 &     0.07 &        0.22 &  0.66 &     0.12 \\
\textbf{PDM1\_g } &        1.00 &  0.00 &     0.00 &        0.81 &  0.15 &     0.04 &        0.39 &  0.50 &     0.11 &        0.21 &  0.62 &     0.17 &        0.13 &  0.69 &     0.18 \\
\textbf{PDM1\_r } &        1.00 &  0.00 &     0.00 &        0.90 &  0.07 &     0.03 &        0.46 &  0.41 &     0.13 &        0.32 &  0.54 &     0.14 &        0.27 &  0.59 &     0.14 \\
\textbf{MHAOV\_g} &        0.91 &  0.07 &     0.02 &        0.56 &  0.44 &     0.00 &        0.24 &  0.76 &     0.00 &        0.18 &  0.82 &     0.00 &        0.17 &  0.83 &     0.00 \\
\textbf{MHAOV\_r} &        0.95 &  0.04 &     0.01 &        0.57 &  0.36 &     0.07 &        0.38 &  0.62 &     0.00 &        0.25 &  0.75 &     0.00 &        0.21 &  0.79 &     0.00 \\
\textbf{QME\_g  } &        0.96 &  0.04 &     0.00 &        0.38 &  0.59 &     0.03 &        0.24 &  0.76 &     0.00 &        0.14 &  0.86 &     0.00 &        0.11 &  0.89 &     0.00 \\
\textbf{QME\_r  } &        0.86 &  0.11 &     0.03 &        0.42 &  0.52 &     0.06 &        0.27 &  0.73 &     0.00 &        0.16 &  0.84 &     0.00 &        0.14 &  0.86 &     0.00 \\
\textbf{QMICS\_g} &        0.95 &  0.04 &     0.01 &        0.44 &  0.53 &     0.03 &        0.23 &  0.76 &     0.01 &        0.13 &  0.86 &     0.01 &        0.13 &  0.87 &     0.00 \\
\textbf{QMICS\_r} &        0.90 &  0.07 &     0.03 &        0.49 &  0.49 &     0.02 &        0.29 &  0.70 &     0.01 &        0.18 &  0.81 &     0.01 &        0.16 &  0.84 &     0.00 \\
\textbf{QMIEU\_g} &        0.91 &  0.08 &     0.01 &        0.37 &  0.59 &     0.04 &        0.21 &  0.79 &     0.00 &        0.12 &  0.88 &     0.00 &        0.10 &  0.90 &     0.00 \\
\textbf{QMIEU\_r} &        0.86 &  0.10 &     0.04 &        0.41 &  0.56 &     0.03 &        0.27 &  0.73 &     0.00 &        0.16 &  0.84 &     0.00 &        0.13 &  0.87 &     0.00 \\
\bottomrule
\end{tabular}}
\end{table}


    \subsection{RRL - Multi band}
    
\begin{table}
\centering
\caption{Resumen de resultados RRL multiband}
\label{RRLmb}
\scalebox{.5}{
\begin{tabular}{lrrrrrrrrrrrrrrr}
\toprule
{} & \multicolumn{3}{l}{10 samples:} & \multicolumn{3}{l}{20 samples:} & \multicolumn{3}{l}{30 samples:} & \multicolumn{3}{l}{40 samples:} & \multicolumn{3}{l}{50 samples:} \\
{} &     Differs &    Ok & Multiply &     Differs &    Ok & Multiply &     Differs &    Ok & Multiply &     Differs &    Ok & Multiply &     Differs &    Ok & Multiply \\
\midrule
\textbf{catalog} &        0.00 &  0.99 &     0.01 &        0.00 &  0.99 &     0.01 &        0.00 &  0.99 &     0.01 &        0.00 &  0.99 &     0.01 &         0.0 &  0.99 &     0.01 \\
\textbf{MHAOV\_g} &        0.73 &  0.27 &     0.00 &        0.24 &  0.75 &     0.01 &        0.15 &  0.85 &     0.00 &        0.09 &  0.91 &     0.00 &         0.1 &  0.90 &     0.00 \\
\textbf{MHAOV\_r} &        0.74 &  0.26 &     0.00 &        0.24 &  0.75 &     0.01 &        0.15 &  0.85 &     0.00 &        0.09 &  0.91 &     0.00 &         0.1 &  0.90 &     0.00 \\
\bottomrule
\end{tabular}}
\end{table}

    \subsection{Ceph - Single band}
    
\begin{table}
\centering
\caption{Resumen de resultados Ceph single band.}
\label{anexo:Cephsb}
\scalebox{.48}{
\begin{tabular}{lrrrrrrrrrrrrrrr}
\toprule
{} & \multicolumn{3}{l}{10 samples:} & \multicolumn{3}{l}{20 samples:} & \multicolumn{3}{l}{30 samples:} & \multicolumn{3}{l}{40 samples:} & \multicolumn{3}{l}{50 samples:} \\
{} &     Differs &    Ok & Multiply &     Differs &    Ok & Multiply &     Differs &    Ok & Multiply &     Differs &    Ok & Multiply &     Differs &    Ok & Multiply \\
\midrule
\textbf{catalog} &        0.14 &  0.82 &     0.04 &        0.14 &  0.84 &     0.02 &        0.14 &  0.84 &     0.02 &        0.14 &  0.84 &     0.02 &        0.14 &  0.84 &     0.02 \\
\textbf{LS\_g   } &        0.76 &  0.22 &     0.02 &        0.25 &  0.74 &     0.01 &        0.18 &  0.79 &     0.03 &        0.11 &  0.89 &     0.00 &        0.08 &  0.92 &     0.00 \\
\textbf{LS\_r   } &        0.77 &  0.21 &     0.02 &        0.44 &  0.54 &     0.02 &        0.23 &  0.74 &     0.03 &        0.20 &  0.80 &     0.00 &        0.12 &  0.88 &     0.00 \\
\textbf{LKSL\_g } &        0.98 &  0.02 &     0.00 &        0.35 &  0.65 &     0.00 &        0.18 &  0.82 &     0.00 &        0.10 &  0.88 &     0.02 &        0.06 &  0.92 &     0.02 \\
\textbf{LKSL\_r } &        0.97 &  0.03 &     0.00 &        0.47 &  0.52 &     0.01 &        0.19 &  0.79 &     0.02 &        0.10 &  0.87 &     0.03 &        0.05 &  0.91 &     0.04 \\
\textbf{PDM1\_g } &        0.99 &  0.01 &     0.00 &        0.50 &  0.46 &     0.04 &        0.17 &  0.78 &     0.05 &        0.08 &  0.87 &     0.05 &        0.06 &  0.93 &     0.01 \\
\textbf{PDM1\_r } &        0.99 &  0.00 &     0.01 &        0.67 &  0.28 &     0.05 &        0.15 &  0.77 &     0.08 &        0.10 &  0.84 &     0.06 &        0.06 &  0.87 &     0.07 \\
\textbf{MHAOV\_g} &        0.77 &  0.23 &     0.00 &        0.24 &  0.74 &     0.02 &        0.19 &  0.81 &     0.00 &        0.12 &  0.88 &     0.00 &        0.08 &  0.92 &     0.00 \\
\textbf{MHAOV\_r} &        0.79 &  0.20 &     0.01 &        0.45 &  0.54 &     0.01 &        0.25 &  0.75 &     0.00 &        0.17 &  0.83 &     0.00 &        0.15 &  0.85 &     0.00 \\
\textbf{QME\_g  } &        0.79 &  0.21 &     0.00 &        0.28 &  0.72 &     0.00 &        0.15 &  0.84 &     0.01 &        0.09 &  0.90 &     0.01 &        0.08 &  0.92 &     0.00 \\
\textbf{QME\_r  } &        0.86 &  0.10 &     0.04 &        0.33 &  0.66 &     0.01 &        0.12 &  0.86 &     0.02 &        0.10 &  0.88 &     0.02 &        0.08 &  0.90 &     0.02 \\
\textbf{QMICS\_g} &        0.87 &  0.12 &     0.01 &        0.38 &  0.61 &     0.01 &        0.24 &  0.75 &     0.01 &        0.10 &  0.90 &     0.00 &        0.08 &  0.92 &     0.00 \\
\textbf{QMICS\_r} &        0.91 &  0.08 &     0.01 &        0.50 &  0.49 &     0.01 &        0.24 &  0.73 &     0.03 &        0.17 &  0.82 &     0.01 &        0.13 &  0.85 &     0.02 \\
\textbf{QMIEU\_g} &        0.81 &  0.18 &     0.01 &        0.36 &  0.64 &     0.00 &        0.16 &  0.83 &     0.01 &        0.08 &  0.92 &     0.00 &        0.07 &  0.93 &     0.00 \\
\textbf{QMIEU\_r} &        0.89 &  0.10 &     0.01 &        0.44 &  0.55 &     0.01 &        0.21 &  0.76 &     0.03 &        0.12 &  0.86 &     0.02 &        0.08 &  0.89 &     0.03 \\
\bottomrule
\end{tabular}}
\end{table}    
    
    \subsection{Ceph - Multi band}
    
\begin{table}
\centering
\caption{Resumen de resultados Ceph multi band.}
\label{anexo:Cephmb}
\scalebox{.5}{
\begin{tabular}{lrrrrrrrrrrrrrrr}
\toprule
{} & \multicolumn{3}{l}{10 samples:} & \multicolumn{3}{l}{20 samples:} & \multicolumn{3}{l}{30 samples:} & \multicolumn{3}{l}{40 samples:} & \multicolumn{3}{l}{50 samples:} \\
{} &     Differs &    Ok & Multiply &     Differs &    Ok & Multiply &     Differs &    Ok & Multiply &     Differs &    Ok & Multiply &     Differs &    Ok & Multiply \\
\midrule
\textbf{catalog} &        0.14 &  0.82 &     0.04 &        0.14 &  0.82 &     0.04 &        0.14 &  0.82 &     0.04 &        0.14 &  0.82 &     0.04 &        0.14 &  0.82 &     0.04 \\
\textbf{MHAOV\_g} &        0.39 &  0.59 &     0.02 &        0.14 &  0.86 &     0.00 &        0.12 &  0.88 &     0.00 &        0.08 &  0.92 &     0.00 &        0.04 &  0.96 &     0.00 \\
\textbf{MHAOV\_r} &        0.39 &  0.59 &     0.02 &        0.14 &  0.86 &     0.00 &        0.12 &  0.88 &     0.00 &        0.08 &  0.92 &     0.00 &        0.04 &  0.96 &     0.00 \\
\bottomrule
\end{tabular}}
\end{table}    
    
    \subsection{LPV - Single band}

\begin{table}
\centering
\caption{Resumen de resultados LPV single band.}
\label{anexo:LPVsb}
\scalebox{.48}{
\begin{tabular}{lrrrrrrrrrrrrrrr}
\toprule
{} & \multicolumn{3}{l}{10 samples:} & \multicolumn{3}{l}{20 samples:} & \multicolumn{3}{l}{30 samples:} & \multicolumn{3}{l}{40 samples:} & \multicolumn{3}{l}{50 samples:} \\
{} &     Differs &    Ok & Multiply &     Differs &    Ok & Multiply &     Differs &    Ok & Multiply &     Differs &    Ok & Multiply &     Differs &    Ok & Multiply \\
\midrule
\textbf{catalog} &        0.17 &  0.80 &     0.03 &        0.17 &  0.80 &     0.03 &        0.17 &  0.80 &     0.03 &        0.17 &  0.79 &     0.04 &        0.17 &  0.79 &     0.04 \\
\textbf{LS\_g   } &        0.69 &  0.25 &     0.06 &        0.44 &  0.50 &     0.06 &        0.38 &  0.55 &     0.07 &        0.31 &  0.59 &     0.10 &        0.26 &  0.67 &     0.07 \\
\textbf{LS\_r   } &        0.62 &  0.35 &     0.03 &        0.39 &  0.60 &     0.01 &        0.32 &  0.65 &     0.03 &        0.27 &  0.71 &     0.02 &        0.24 &  0.74 &     0.02 \\
\textbf{LKSL\_g } &        0.97 &  0.03 &     0.00 &        0.60 &  0.34 &     0.06 &        0.29 &  0.55 &     0.16 &        0.24 &  0.58 &     0.18 &        0.16 &  0.56 &     0.28 \\
\textbf{LKSL\_r } &        0.96 &  0.04 &     0.00 &        0.47 &  0.45 &     0.08 &        0.36 &  0.52 &     0.12 &        0.23 &  0.65 &     0.12 &        0.11 &  0.69 &     0.20 \\
\textbf{PDM1\_g } &        0.85 &  0.07 &     0.08 &        0.56 &  0.23 &     0.21 &        0.22 &  0.45 &     0.33 &        0.17 &  0.56 &     0.27 &        0.13 &  0.58 &     0.29 \\
\textbf{PDM1\_r } &        0.88 &  0.06 &     0.06 &        0.57 &  0.27 &     0.16 &        0.22 &  0.62 &     0.16 &        0.16 &  0.71 &     0.13 &        0.10 &  0.74 &     0.16 \\
\textbf{MHAOV\_g} &        0.65 &  0.30 &     0.05 &        0.39 &  0.55 &     0.06 &        0.28 &  0.63 &     0.09 &        0.24 &  0.63 &     0.13 &        0.22 &  0.70 &     0.08 \\
\textbf{MHAOV\_r} &        0.63 &  0.35 &     0.02 &        0.36 &  0.62 &     0.02 &        0.30 &  0.66 &     0.04 &        0.23 &  0.73 &     0.04 &        0.23 &  0.75 &     0.02 \\
\textbf{QME\_g  } &        0.96 &  0.03 &     0.01 &        0.61 &  0.25 &     0.14 &        0.42 &  0.43 &     0.15 &        0.34 &  0.42 &     0.24 &        0.28 &  0.47 &     0.25 \\
\textbf{QME\_r  } &        0.93 &  0.06 &     0.01 &        0.59 &  0.27 &     0.14 &        0.49 &  0.29 &     0.22 &        0.42 &  0.36 &     0.22 &        0.40 &  0.36 &     0.24 \\
\textbf{QMICS\_g} &        0.97 &  0.01 &     0.02 &        0.82 &  0.11 &     0.07 &        0.56 &  0.26 &     0.18 &        0.49 &  0.30 &     0.21 &        0.44 &  0.31 &     0.25 \\
\textbf{QMICS\_r} &        0.99 &  0.00 &     0.01 &        0.78 &  0.11 &     0.11 &        0.65 &  0.17 &     0.18 &        0.54 &  0.26 &     0.20 &        0.52 &  0.25 &     0.23 \\
\textbf{QMIEU\_g} &        0.91 &  0.04 &     0.05 &        0.65 &  0.22 &     0.13 &        0.40 &  0.42 &     0.18 &        0.33 &  0.44 &     0.23 &        0.32 &  0.51 &     0.17 \\
\textbf{QMIEU\_r} &        0.95 &  0.04 &     0.01 &        0.66 &  0.21 &     0.13 &        0.42 &  0.33 &     0.25 &        0.38 &  0.40 &     0.22 &        0.38 &  0.41 &     0.21 \\
\bottomrule
\end{tabular}}
\end{table}    
    
    
    \subsection{LPV - Multi band}
\begin{table}
\centering
\caption{Resumen de resultados LPV multi band.}
\label{anexo:LPVmb}
\scalebox{.5}{
\begin{tabular}{lrrrrrrrrrrrrrrr}
\toprule
{} & \multicolumn{3}{l}{10 samples:} & \multicolumn{3}{l}{20 samples:} & \multicolumn{3}{l}{30 samples:} & \multicolumn{3}{l}{40 samples:} & \multicolumn{3}{l}{50 samples:} \\
{} &     Differs &    Ok & Multiply &     Differs &    Ok & Multiply &     Differs &    Ok & Multiply &     Differs &    Ok & Multiply &     Differs &    Ok & Multiply \\
\midrule
\textbf{catalog} &        0.17 &  0.80 &     0.03 &        0.17 &  0.80 &     0.03 &        0.17 &  0.80 &     0.03 &        0.17 &  0.80 &     0.03 &        0.17 &  0.80 &     0.03 \\
\textbf{MHAOV\_g} &        0.42 &  0.53 &     0.05 &        0.21 &  0.77 &     0.02 &        0.19 &  0.81 &     0.00 &        0.20 &  0.77 &     0.03 &        0.20 &  0.79 &     0.01 \\
\textbf{MHAOV\_r} &        0.42 &  0.53 &     0.05 &        0.22 &  0.76 &     0.02 &        0.19 &  0.81 &     0.00 &        0.20 &  0.77 &     0.03 &        0.20 &  0.79 &     0.01 \\
\bottomrule
\end{tabular}}
\end{table}    
    
    
    \subsection{DSCT - Single band}
    
\begin{table}
\centering
\caption{Resumen de resultados DSCT single band.}
\label{anexo:DSCTsb}
\scalebox{.48}{
\begin{tabular}{lrrrrrrrrrrrrrrr}
\toprule
{} & \multicolumn{3}{l}{10 samples:} & \multicolumn{3}{l}{20 samples:} & \multicolumn{3}{l}{30 samples:} & \multicolumn{3}{l}{40 samples:} & \multicolumn{3}{l}{50 samples:} \\
{} &     Differs &    Ok & Multiply &     Differs &    Ok & Multiply &     Differs &    Ok & Multiply &     Differs &    Ok & Multiply &     Differs &    Ok & Multiply \\
\midrule
\textbf{catalog} &        0.11 &  0.89 &      0.0 &        0.11 &  0.89 &     0.00 &        0.11 &  0.89 &     0.00 &        0.11 &  0.89 &     0.00 &        0.11 &  0.89 &     0.00 \\
\textbf{LS\_g   } &        0.98 &  0.02 &      0.0 &        0.84 &  0.16 &     0.00 &        0.72 &  0.28 &     0.00 &        0.67 &  0.33 &     0.00 &        0.59 &  0.41 &     0.00 \\
\textbf{LS\_r   } &        1.00 &  0.00 &      0.0 &        0.87 &  0.13 &     0.00 &        0.72 &  0.28 &     0.00 &        0.68 &  0.32 &     0.00 &        0.54 &  0.45 &     0.01 \\
\textbf{LKSL\_g } &        0.99 &  0.01 &      0.0 &        0.92 &  0.08 &     0.00 &        0.74 &  0.17 &     0.09 &        0.61 &  0.18 &     0.21 &        0.46 &  0.18 &     0.36 \\
\textbf{LKSL\_r } &        1.00 &  0.00 &      0.0 &        0.89 &  0.11 &     0.00 &        0.81 &  0.16 &     0.03 &        0.71 &  0.15 &     0.14 &        0.55 &  0.20 &     0.25 \\
\textbf{PDM1\_g } &        1.00 &  0.00 &      0.0 &        0.98 &  0.01 &     0.01 &        0.82 &  0.14 &     0.04 &        0.60 &  0.19 &     0.21 &        0.55 &  0.22 &     0.23 \\
\textbf{PDM1\_r } &        0.99 &  0.01 &      0.0 &        0.99 &  0.00 &     0.01 &        0.80 &  0.14 &     0.06 &        0.74 &  0.15 &     0.11 &        0.56 &  0.23 &     0.21 \\
\textbf{MHAOV\_g} &        0.98 &  0.02 &      0.0 &        0.81 &  0.19 &     0.00 &        0.74 &  0.26 &     0.00 &        0.68 &  0.32 &     0.00 &        0.50 &  0.50 &     0.00 \\
\textbf{MHAOV\_r} &        1.00 &  0.00 &      0.0 &        0.84 &  0.16 &     0.00 &        0.78 &  0.22 &     0.00 &        0.70 &  0.30 &     0.00 &        0.51 &  0.49 &     0.00 \\
\textbf{QME\_g  } &        0.98 &  0.02 &      0.0 &        0.76 &  0.22 &     0.02 &        0.66 &  0.34 &     0.00 &        0.55 &  0.43 &     0.02 &        0.36 &  0.62 &     0.02 \\
\textbf{QME\_r  } &        0.98 &  0.02 &      0.0 &        0.77 &  0.23 &     0.00 &        0.70 &  0.30 &     0.00 &        0.57 &  0.43 &     0.00 &        0.40 &  0.58 &     0.02 \\
\textbf{QMICS\_g} &        0.99 &  0.01 &      0.0 &        0.86 &  0.13 &     0.01 &        0.75 &  0.24 &     0.01 &        0.56 &  0.42 &     0.02 &        0.45 &  0.53 &     0.02 \\
\textbf{QMICS\_r} &        1.00 &  0.00 &      0.0 &        0.84 &  0.16 &     0.00 &        0.71 &  0.29 &     0.00 &        0.56 &  0.44 &     0.00 &        0.40 &  0.60 &     0.00 \\
\textbf{QMIEU\_g} &        1.00 &  0.00 &      0.0 &        0.83 &  0.16 &     0.01 &        0.69 &  0.30 &     0.01 &        0.54 &  0.46 &     0.00 &        0.37 &  0.60 &     0.03 \\
\textbf{QMIEU\_r} &        0.99 &  0.01 &      0.0 &        0.79 &  0.21 &     0.00 &        0.70 &  0.30 &     0.00 &        0.58 &  0.42 &     0.00 &        0.42 &  0.58 &     0.00 \\
\bottomrule
\end{tabular}}
\end{table}    
    
    
    \subsection{DSCT - Multi band}
    
\begin{table}
\centering
\caption{Resumen de resultados DSCT multi band.}
\label{anexo:DSCTmb}
\scalebox{.5}{
\begin{tabular}{lrrrrrrrrrrrrrrr}
\toprule
{} & \multicolumn{3}{l}{10 samples:} & \multicolumn{3}{l}{20 samples:} & \multicolumn{3}{l}{30 samples:} & \multicolumn{3}{l}{40 samples:} & \multicolumn{3}{l}{50 samples:} \\
{} &     Differs &    Ok & Multiply &     Differs &    Ok & Multiply &     Differs &    Ok & Multiply &     Differs &    Ok & Multiply &     Differs &    Ok & Multiply \\
\midrule
\textbf{catalog} &        0.11 &  0.89 &      0.0 &        0.11 &  0.89 &      0.0 &        0.11 &  0.89 &      0.0 &        0.11 &  0.89 &      0.0 &        0.11 &  0.89 &      0.0 \\
\textbf{MHAOV\_g} &        0.88 &  0.12 &      0.0 &        0.67 &  0.33 &      0.0 &        0.52 &  0.48 &      0.0 &        0.50 &  0.50 &      0.0 &        0.46 &  0.54 &      0.0 \\
\textbf{MHAOV\_r} &        0.88 &  0.12 &      0.0 &        0.67 &  0.33 &      0.0 &        0.52 &  0.48 &      0.0 &        0.49 &  0.51 &      0.0 &        0.46 &  0.54 &      0.0 \\
\bottomrule
\end{tabular}}
\end{table}    
    
    
    \subsection{EB - Single band}
    
\begin{table}
\centering
\caption{Resumen de resultados EB single band.}
\label{anexo:EBsb}
\scalebox{.4}{
\begin{tabular}{lrrrrrrrrrrrrrrrrrrrr}
\toprule
{} & \multicolumn{4}{l}{10 samples:} & \multicolumn{4}{l}{20 samples:} & \multicolumn{4}{l}{30 samples:} & \multicolumn{4}{l}{40 samples:} & \multicolumn{4}{l}{50 samples:} \\
{} &     Differs &    Ok &  Half & Multiply &     Differs &    Ok &  Half & Multiply &     Differs &    Ok &  Half & Multiply &     Differs &    Ok &  Half & Multiply &     Differs &    Ok &  Half & Multiply \\
\midrule
\textbf{catalog} &        0.01 &  0.97 &  0.02 &      0.0 &        0.01 &  0.97 &  0.02 &     0.00 &        0.01 &  0.97 &  0.02 &     0.00 &        0.01 &  0.97 &  0.02 &     0.00 &        0.01 &  0.97 &  0.02 &     0.00 \\
\textbf{LS\_g   } &        0.89 &  0.00 &  0.11 &      0.0 &        0.64 &  0.02 &  0.34 &     0.00 &        0.61 &  0.02 &  0.37 &     0.00 &        0.56 &  0.02 &  0.42 &     0.00 &        0.50 &  0.02 &  0.48 &     0.00 \\
\textbf{LS\_r   } &        0.87 &  0.02 &  0.11 &      0.0 &        0.66 &  0.01 &  0.33 &     0.00 &        0.57 &  0.01 &  0.42 &     0.00 &        0.55 &  0.01 &  0.44 &     0.00 &        0.51 &  0.01 &  0.48 &     0.00 \\
\textbf{LKSL\_g } &        0.98 &  0.00 &  0.02 &      0.0 &        0.78 &  0.01 &  0.21 &     0.00 &        0.55 &  0.13 &  0.27 &     0.05 &        0.36 &  0.29 &  0.29 &     0.06 &        0.30 &  0.31 &  0.30 &     0.09 \\
\textbf{LKSL\_r } &        1.00 &  0.00 &  0.00 &      0.0 &        0.83 &  0.01 &  0.16 &     0.00 &        0.62 &  0.10 &  0.28 &     0.00 &        0.43 &  0.21 &  0.29 &     0.07 &        0.26 &  0.29 &  0.36 &     0.09 \\
\textbf{PDM1\_g } &        1.00 &  0.00 &  0.00 &      0.0 &        0.97 &  0.01 &  0.02 &     0.00 &        0.52 &  0.19 &  0.24 &     0.05 &        0.37 &  0.21 &  0.33 &     0.09 &        0.31 &  0.26 &  0.35 &     0.08 \\
\textbf{PDM1\_r } &        1.00 &  0.00 &  0.00 &      0.0 &        0.91 &  0.02 &  0.07 &     0.00 &        0.62 &  0.08 &  0.24 &     0.06 &        0.46 &  0.17 &  0.33 &     0.04 &        0.32 &  0.24 &  0.37 &     0.07 \\
\textbf{MHAOV\_g} &        0.89 &  0.00 &  0.11 &      0.0 &        0.61 &  0.02 &  0.37 &     0.00 &        0.61 &  0.02 &  0.37 &     0.00 &        0.56 &  0.02 &  0.41 &     0.01 &        0.49 &  0.03 &  0.47 &     0.01 \\
\textbf{MHAOV\_r} &        0.90 &  0.01 &  0.09 &      0.0 &        0.61 &  0.01 &  0.38 &     0.00 &        0.59 &  0.01 &  0.40 &     0.00 &        0.56 &  0.01 &  0.43 &     0.00 &        0.51 &  0.01 &  0.47 &     0.01 \\
\textbf{QME\_g  } &        0.95 &  0.00 &  0.05 &      0.0 &        0.67 &  0.01 &  0.32 &     0.00 &        0.60 &  0.03 &  0.37 &     0.00 &        0.53 &  0.02 &  0.45 &     0.00 &        0.44 &  0.02 &  0.54 &     0.00 \\
\textbf{QME\_r  } &        0.91 &  0.01 &  0.08 &      0.0 &        0.69 &  0.02 &  0.29 &     0.00 &        0.62 &  0.04 &  0.34 &     0.00 &        0.53 &  0.04 &  0.43 &     0.00 &        0.48 &  0.02 &  0.50 &     0.00 \\
\textbf{QMICS\_g} &        0.97 &  0.00 &  0.03 &      0.0 &        0.76 &  0.02 &  0.22 &     0.00 &        0.62 &  0.02 &  0.36 &     0.00 &        0.54 &  0.05 &  0.41 &     0.00 &        0.52 &  0.03 &  0.45 &     0.00 \\
\textbf{QMICS\_r} &        0.98 &  0.00 &  0.02 &      0.0 &        0.76 &  0.01 &  0.23 &     0.00 &        0.61 &  0.02 &  0.37 &     0.00 &        0.55 &  0.06 &  0.39 &     0.00 &        0.55 &  0.03 &  0.42 &     0.00 \\
\textbf{QMIEU\_g} &        0.93 &  0.00 &  0.07 &      0.0 &        0.72 &  0.02 &  0.26 &     0.00 &        0.60 &  0.03 &  0.37 &     0.00 &        0.53 &  0.04 &  0.43 &     0.00 &        0.50 &  0.02 &  0.48 &     0.00 \\
\textbf{QMIEU\_r} &        0.92 &  0.00 &  0.08 &      0.0 &        0.72 &  0.01 &  0.26 &     0.01 &        0.59 &  0.03 &  0.38 &     0.00 &        0.49 &  0.06 &  0.45 &     0.00 &        0.52 &  0.01 &  0.47 &     0.00 \\
\bottomrule
\end{tabular}}
\end{table}    
    
    
    \subsection{EB - Multi band}

\begin{table}
\centering
\caption{Resumen de resultados EB multi band.}
\label{anexo:EBmb}
\scalebox{.5}{
\begin{tabular}{lrrrrrrrrrrrrrrrrrrrr}
\toprule
{} & \multicolumn{4}{l}{10 samples:} & \multicolumn{4}{l}{20 samples:} & \multicolumn{4}{l}{30 samples:} & \multicolumn{4}{l}{40 samples:} & \multicolumn{4}{l}{50 samples:} \\
{} &     Differs &    Ok &  Half & Multiply &     Differs &    Ok &  Half & Multiply &     Differs &    Ok &  Half & Multiply &     Differs &    Ok &  Half & Multiply &     Differs &    Ok &  Half & Multiply \\
\midrule
\textbf{catalog} &        0.01 &  0.97 &  0.02 &      0.0 &        0.01 &  0.97 &  0.02 &      0.0 &        0.01 &  0.97 &  0.02 &      0.0 &        0.01 &  0.97 &  0.02 &      0.0 &        0.01 &  0.97 &  0.02 &      0.0 \\
\textbf{MHAOV\_g} &        0.71 &  0.02 &  0.27 &      0.0 &        0.55 &  0.02 &  0.43 &      0.0 &        0.47 &  0.02 &  0.51 &      0.0 &        0.48 &  0.02 &  0.50 &      0.0 &        0.42 &  0.02 &  0.56 &      0.0 \\
\textbf{MHAOV\_r} &        0.71 &  0.02 &  0.27 &      0.0 &        0.55 &  0.02 &  0.43 &      0.0 &        0.47 &  0.02 &  0.51 &      0.0 &        0.48 &  0.02 &  0.50 &      0.0 &        0.42 &  0.02 &  0.56 &      0.0 \\
\bottomrule
\end{tabular}}
\end{table}


\end{anexo}
% FIN DEL DOCUMENTO
\end{document}
